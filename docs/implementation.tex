\noindent
\begin{lstlisting}
// Program name: cylon.cpp
// Author:       Massimo Di Pierro
// License:      BSD
\end{lstlisting}
\noindent
Import the necessary libraries: \begin{lstlisting}
#include "fstream"
#include <sstream>
#include <string>
#include <iostream>
#include "math.h"
#include "stdlib.h"
#include "vector"
#include "set"
#if defined(_MSC_VER)
#include <gl/glut.h>
#else
#include <GLUT/glut.h>
#endif
using namespace std;
\end{lstlisting}
\noindent
Define useful macros and constants: \begin{lstlisting}
#define self (*this)
#define array vector // to avoid name collions
#define forXYZ(i) for(int i=0; i<3; i++)
#define forEach(i,s) for(i=(s).begin();i!=(s).end();i++)
#define OBJ(iterator) (*(*iterator))
const float Pi = 3.1415926535897931;
const float PRECISION = 0.00001;
const int X = 0;
const int Y = 1;
const int Z = 2;
const float DT  = 0.017f; // Hard-coded dt for Universe::evolve()
const int MSPF = 17; // msec per frame: glutTimerFunc() only takes integers for params
                     // lower value = higher frame rate


float uniform(float a=0, float b=1) {
  int n = 10000;
  return a+(b-a)*(float)(rand() % n)/n;
}
\end{lstlisting}
\noindent
Define class vector: \begin{lstlisting}
class Vector {
public:
  float v[3];
  Vector(float x=0, float y=0, float z=0) {
    v[X] = x; v[Y] = y; v[Z] = z;
  }
  float operator()(int i) const { return v[i]; }
  float &operator()(int i) { return v[i]; }
};
\end{lstlisting}
\noindent
Define operations between vectors: \begin{lstlisting}
Vector operator*(float c, const Vector &v) {
  return Vector(c*v(X),c*v(Y),c*v(Z));
}
Vector operator*(const Vector &v, float c) {
  return c*v;
}
Vector operator/(const Vector &v, float c) {
  return (1.0/c)*v;
}
Vector operator+(const Vector &v, const Vector &w) {
  return Vector(v(X)+w(X),v(Y)+w(Y),v(Z)+w(Z));
}
Vector operator-(const Vector &v, const Vector &w) {
  return Vector(v(X)-w(X),v(Y)-w(Y),v(Z)-w(Z));
}
float operator*(const Vector &v, const Vector &w) {
  return v(X)*w(X) + v(Y)*w(Y) + v(Z)*w(Z);
}
float norm(const Vector &v) {
  return sqrt(v*v);
}
Vector versor(const Vector &v) {
  float d = norm(v);
  return (d>0)?(v/d):v;
}
Vector cross(const Vector &v, const Vector &w) {
  return Vector(v(Y)*w(Z)-v(Z)*w(Y),
                v(Z)*w(X)-v(X)*w(Z),
                v(X)*w(Y)-v(Y)*w(X));
}
\end{lstlisting}
\noindent
Class matrix is a base for rotation and inertiatensor: \begin{lstlisting}
class Matrix {
public:
  float m[3][3];
  Matrix() { forXYZ(i) forXYZ(j) m[i][j] = 0; }
  const float operator()(int i, int j) const { return m[i][j]; }
  float &operator()(int i, int j) { return m[i][j]; }
  Matrix t();
};
\end{lstlisting}
\noindent
Operations between matrices: \begin{lstlisting}
Vector operator*(const Matrix &R, const Vector &v) {
  return Vector(R(X,X)*v(X)+R(X,Y)*v(Y)+R(X,Z)*v(Z),
                 R(Y,X)*v(X)+R(Y,Y)*v(Y)+R(Y,Z)*v(Z),
                 R(Z,X)*v(X)+R(Z,Y)*v(Y)+R(Z,Z)*v(Z));
}

Matrix operator*(const Matrix &R, const Matrix &S) {
  Matrix T;
  forXYZ(i) forXYZ(j) forXYZ(k) T(i,j) += R(i,k)*S(k,j);
  return T;
}

float det(const Matrix &R) {
  return R(X,X)*(R(Z,Z)*R(Y,Y)-R(Z,Y)*R(Y,Z))
    -R(Y,X)*(R(Z,Z)*R(X,Y)-R(Z,Y)*R(X,Z))
    +R(Z,X)*(R(Y,Z)*R(X,Y)-R(Y,Y)*R(X,Z));
}

Matrix operator/(float c, const Matrix &R) {
  Matrix T;
  float d = c/det(R);
  T(X,X) = (R(Z,Z)*R(Y,Y)-R(Z,Y)*R(Y,Z))*d;
  T(X,Y) = (R(Z,Y)*R(X,Z)-R(Z,Z)*R(X,Y))*d;
  T(X,Z) = (R(Y,Z)*R(X,Y)-R(Y,Y)*R(X,Z))*d;
  T(Y,X) = (R(Z,X)*R(Y,Z)-R(Z,Z)*R(Y,X))*d;
  T(Y,Y) = (R(Z,Z)*R(X,X)-R(Z,X)*R(X,Z))*d;
  T(Y,Z) = (R(Y,X)*R(X,Z)-R(Y,Z)*R(X,X))*d;
  T(Z,X) = (R(Z,Y)*R(Y,X)-R(Z,X)*R(Y,Y))*d;
  T(Z,Y) = (R(Z,Y)*R(X,Y)-R(Z,Y)*R(X,X))*d;
  T(Z,Z) = (R(Y,Y)*R(X,X)-R(Y,X)*R(X,Y))*d;
  return T;
}

Matrix Matrix::t() {
  Matrix Mt;
  forXYZ(j) forXYZ(k) Mt(j,k)=self(k,j);
  return Mt;
}
\end{lstlisting}
\noindent
Class rotation is a matrix: \begin{lstlisting}
class Rotation : public Matrix {
public:
  Rotation() { forXYZ(i) forXYZ(j) m[i][j] = (i==j)?1:0; }
  Rotation(const Vector& v) {
    float theta = norm(v);
    if(theta<PRECISION) {
      forXYZ(i) forXYZ(j) m[i][j] = (i==j)?1:0;
    } else {
      float s = sin(theta), c = cos(theta);
      float t = 1-c;
      float x = v(X)/theta, y = v(Y)/theta, z = v(Z)/theta;
      m[X][X] = t*x*x+c;   m[X][Y] = t*x*y-s*z; m[X][Z] = t*x*z+s*y;
      m[Y][X] = t*x*y+s*z; m[Y][Y] = t*y*y+c;   m[Y][Z] = t*y*z-s*x;
      m[Z][X] = t*x*z-s*y; m[Z][Y] = t*y*z+s*x; m[Z][Z] = t*z*z+c;
    }
  }
};
\end{lstlisting}
\noindent
Class inertiatensor is also a matrix: \begin{lstlisting}
class InertiaTensor : public Matrix {};
InertiaTensor operator+(const InertiaTensor &a, const InertiaTensor &b) {
  InertiaTensor c = a;
  forXYZ(i) forXYZ(j) c(i,j) += b(i,j);
  return c;
}
\end{lstlisting}
\noindent
Class body (describes a rigid body object): \begin{lstlisting}
class Body {
public:
  // object shape
  float radius;
  array<Vector> r;   // vertices in local coordinates
  array<array<int> > faces;
  Vector color;      // color of the object;
  bool visible;
  // properties of the body /////////////////////////
  bool locked;       // if set true, don't integrate
  float m;           // mass
  InertiaTensor I;   // moments of inertia (3x3 matrix)
  // state of the body //////////////////////////////
  Vector p;          // position of the center of mass
  Vector K;          // momentum
  Matrix R;          // orientation
  Vector L;          // angular momentum
  // auxiliary variables ///////////////////////////
  Matrix inv_I;      // 1/I
  Vector F;
  Vector tau;
  Vector v;          // velocity
  Vector omega;      // angular velocity
  array<Vector> Rr; // rotated r's.
  array<Vector> vertices; // rotated and shifted r's
  // forces and constraints
  // ...
  Body(float m=1.0, float radius=0.2, bool locked=false) {
    this->m = m;
    this->radius = radius;
    this->locked = locked;
    this->R = Rotation();
    I(X,X)=I(Y,Y)=I(Z,Z)=m;
    this->inv_I = 1.0/I;
    this->r.push_back(Vector(0,0,0)); // object contains one point
    this->color = Vector(1,0,0);
    this->visible = true;
    this->update_vertices();
  }
  void clear() { F(X)=F(Y)=F(Z)=tau(X)=tau(Y)=tau(Z)=0; }
  void update_vertices();
  void integrator(float dt);
  void loadObj(const string & file, float scale);
  void draw();
};
\end{lstlisting}
\noindent
Rotate and shift all vertices from local to universe: \begin{lstlisting}
void Body::update_vertices() {
  if(Rr.size()!=r.size()) Rr.resize(r.size());
  if(vertices.size()!=r.size()) vertices.resize(r.size());
  for(int i=0; i<r.size(); i++) {
     Rr[i] = R*r[i];
     vertices[i]=Rr[i]+p;
  }
}
\end{lstlisting}
\noindent
Euler integrator: \begin{lstlisting}
void Body::integrator(float dt) {
  v     = (1.0/m)*K;
  omega = R*(inv_I*(R.t()*L));     // R*inv_I*R.t() in rotated frame
  p     = p + v*dt;                // shift
  K     = K + F*dt;                // push
  R     = Rotation(omega*dt)*R;    // rotate
  L     = L + tau*dt;              // spin
  update_vertices();
}
\end{lstlisting}
\noindent
Interface for all forces.
the constructor can be specific of the force.
the apply methods adds the contribution to f and tau: \begin{lstlisting}
class Force {
public:
  virtual void apply(float dt)=0;
  virtual void draw() {};
};
\end{lstlisting}
\noindent
Gravity is a force: \begin{lstlisting}
class GravityForce : public Force {
public:
  Body *body;
  float g;
  GravityForce(Body *body, float g = 0.01) {
    this->body = body; this->g = g;
  }
  void apply(float dt) {
    body->F(Y) -= (body->m)*g;
  }
};
\end{lstlisting}
\noindent
Spring is a force: \begin{lstlisting}
class SpringForce : public Force {
public:
  Body *bodyA;
  Body *bodyB;
  int iA, iB;
  float kappa, L;
  SpringForce(Body *bodyA, int iA, Body *bodyB, int iB,
              float kappa, float L) {
    this->bodyA = bodyA; this->iA = iA;
    this->bodyB = bodyB; this->iB = iB;
    this->kappa = kappa; this->L = L;
  }
  void apply(float dt) {
    Vector d = ((iB<0)?(bodyB->p):(bodyB->vertices[iB]))-
      ((iA<0)?(bodyA->p):(bodyA->vertices[iA]));
    float n = norm(d);
    if(n>PRECISION) {
      Vector F = kappa*(n-L)*(d/n);
      bodyA->F = bodyA->F+F;
      bodyB->F = bodyB->F-F;
      if(iA>=0)
        bodyA->tau = bodyA->tau + cross(bodyA->Rr[iA],F);
      if(iB>=0)
        bodyB->tau = bodyB->tau - cross(bodyB->Rr[iB],F);
    }
  }
  void draw();
};
\end{lstlisting}
\noindent
A spring can be anchored to a pin: \begin{lstlisting}
class AnchoredSpringForce : public Force {
public:
  Body *body;
  int i;
  Vector pin;
  float kappa, L;
  AnchoredSpringForce(Body *body, int i, Vector pin,
                      float kappa, float L) {
    this->body = body; this->i = i;
    this->pin = pin;
    this->kappa = kappa; this->L = L;
  }
  void apply(float dt) {
    Vector d = pin - ((i<0)?(body->p):(body->vertices[i]));
    float n = norm(d);
    Vector F = kappa*(n-L)*d/n;
    body->F = body->F+F;
    if(i>=0)
      body->tau = body->tau + cross(body->Rr[i],F);
  }
};
\end{lstlisting}
\noindent
Friction is alos a force
(this ignores the shape of the body, assumes a sphere): \begin{lstlisting}
class FrictionForce: public Force {
public:
  Body *body;
  float gamma;
    FrictionForce(Body *body, float gamma) {
      this->body = body; this->gamma = gamma;
    }
    void apply(float dt) {
      body->F = body->F-gamma*(body->v)*dt; // ignores shape
    }
};
\end{lstlisting}
\noindent
Class water, one instance floods the universe: \begin{lstlisting}
class Water: public Force {
public:
  float level, wave, speed;
  float m[41][41];
  float t, x0,dx;
  set<Body*> *bodies;
  Water(set<Body*> *bodies, float level,
        float wave=0.2, float speed=0.2) {
    this->bodies = bodies;
    this->level = level; this->wave = wave, this->speed = speed;
    t = 0; x0 = 5.0; dx = 0.25;
  }
  void apply(float dt) {
    set<Body*>::iterator ibody;
    t = t+dt;
    for(int i=0; i<41; i++)
      for(int j=0; j<41; j++)
        this->m[i][j] = level+wave/2*sin(speed*t+i)+wave/2*sin(0.5*i*j);
    forEach(ibody,*bodies) {
      Body &body = OBJ(ibody); // dereference
      int i = (body.p(X)+x0)/dx;
      int j = (body.p(Z)+x0)/dx;
      if(body.p(Y)<m[i][j]) {
        body.F = body.F + (Vector(0,1,0)-2.0*(body.v))*dt;
        body.L = (1.0-dt)*body.L;
      }
    }
  }
  void draw();
};

Vector resolve_collision(Body &A, Body &B, Vector q, Vector n, 
			 float c, float cf) {
  Vector r_cA = q-A.p;
  Vector r_cB = q-B.p;
  Vector v_cA = cross(A.omega,r_cA)+A.v;
  Vector v_cB = cross(B.omega,r_cB)+B.v;
  Vector v_closing = v_cB-v_cA;
  Vector t = versor(cross(n,v_closing));
  Vector Jo,Jp,J;
  Jo = (c+1)*(v_closing*n)/
    (1.0/A.m+1.0/B.m+(A.inv_I*cross(cross(r_cA,n),r_cA)+
		      B.inv_I*cross(cross(r_cB,n),r_cB))*n)*n;
  Jp = (cf+1)*(v_closing*t)/
    (1.0/A.m+1.0/B.m+(A.inv_I*cross(cross(r_cA,t),r_cA)+
		      B.inv_I*cross(cross(r_cB,t),r_cB))*t)*t;
  J = Jo+Jp;
  A.K = A.K + J;
  B.K = B.K - J;
  A.L = A.L + cross(r_cA,J);
  B.L = B.L - cross(r_cB,J);
}
\end{lstlisting}
\noindent
A constraint has detect and resolve methods: \begin{lstlisting}
class Constraint {
public:
  virtual bool detect()=0;
  virtual void resolve(float dt)=0;
  virtual void draw() {}
};
\end{lstlisting}
\noindent
Class to deal with collision with static plane
(ignore rotation): \begin{lstlisting}
class PlaneConstraint: public Constraint {
public:
  Body *body;
  Vector n,d;
  float penetration, restitution;
  // d is the discance of the plane from origin
  // n is a versor orthogonal to plane
  // (in opposite direction from collision)
  PlaneConstraint(Body *body, float restitution,
                  const Vector &d, const Vector &n) {
    this->body = body;
    this->n = n; this->d = d;
    this->restitution = restitution;
  }
  bool detect() {
    penetration = body->p*n-d*n + body->radius;
    return penetration>=0;
  }
  void resolve(float dt) {
    // move the object back is stuck on plane
    body->p = body->p - penetration*n;
    float K_ortho = n*body->K;
    // optional, deal with friction
    Vector L_ortho = -(body->radius)*cross(n,body->K-K_ortho);
    body->L = (n*body->L)*n + L_ortho;
    // reverse momentum
    if(K_ortho>0)
      body->K = body->K - (restitution+1)*(K_ortho)*n;
  }
};
\end{lstlisting}
\noindent
Default all-2-all collision handler: \begin{lstlisting}
class All2AllCollisions : public Constraint {
public:
  float restitution;
  set<Body*> *bodies;
  All2AllCollisions(set<Body*> *bodies, float c=0.5) {
    this->bodies=bodies;
    restitution = c;
  }
  bool detect() { return true; }
  void resolve(float dt) {
    set<Body*>::iterator ibodyA, ibodyB;
    forEach(ibodyA,*bodies) {
      forEach(ibodyB,*bodies) {
	if((*ibodyA)<(*ibodyB)) {
	  Body &A = OBJ(ibodyA); // dereference
	  Body &B = OBJ(ibodyB); // dereference
	  Vector d = B.p-A.p;
	  Vector n = d/norm(d);
	  Vector v_closing = B.v-A.v;
	  float penetration = A.radius+B.radius-norm(d);
	  if(penetration>0 && v_closing*n<0) {
	    // move bodies to eliminate penetration
	    Vector delta = (penetration/(A.m+B.m))*versor(d);
	    A.p = A.p-B.m*delta;
	    B.p = B.p+A.m*delta;
	    // compute cotact point
	    Vector n = versor(B.p-A.p);
	    Vector q = A.p + A.radius*n;
	    resolve_collision(A,B,q,n,restitution,restitution);
	  }
        }
      }
    }
  }
};
\end{lstlisting}
\noindent
A universe stores bodies, forces, constraints
and evolves in time: \begin{lstlisting}
class Universe {
public:
  float dt;
  // universe state
  set<Body*> bodies;
  set<Force*> forces;
  set<Constraint*> constraints;
  // useful iterators
  set<Body*>::iterator ibody;
  set<Force*>::iterator iforce;
  set<Constraint*>::iterator iconstraint;
  int frame;
  Universe() {
    frame = 0;
  }
  ~Universe() {
    forEach(ibody,bodies) delete (*ibody);
    forEach(iforce,forces) delete (*iforce);
    forEach(iconstraint,constraints) delete (*iconstraint);
  }
  // evolve universe
  void evolve() {
    // clear forces and troques
    forEach(ibody,bodies)
      OBJ(ibody).clear();
    // compute forces and torques
    forEach(iforce,forces)
      OBJ(iforce).apply(dt);
    callback();
    // integrate
    forEach(ibody,bodies)
      if(!OBJ(ibody).locked)
             OBJ(ibody).integrator(dt);
    // handle collisions (not quite right yet)
    forEach(iconstraint,constraints)
      if(OBJ(iconstraint).detect())
             OBJ(iconstraint).resolve(dt);
    frame++;
  }
public:
  virtual void build_universe() { bodies.insert(new Body()); };
  virtual void callback() {};
  virtual void mouse(int button, int state, int x, int y) {};
  virtual void keyboard(unsigned char key, int x, int y) {};
};
\end{lstlisting}
\noindent
Auxiliary functions translate moments of inertia: \begin{lstlisting}
InertiaTensor dI(float m, const Vector &r) {
  InertiaTensor I;
  float r2 = r*r;
  forXYZ(j) forXYZ(k) I(j,k) = m*(((j==k)?r2:0)-r(j)*r(k));
  return I;
}
\end{lstlisting}
\noindent
Auxliary function to include to merge two bodies
needs some more work...: \begin{lstlisting}
Body operator+(const Body &a, const Body &b) {
  Body c;

  c.color = 0.5*(a.color+b.color);
  c.radius = max(a.radius+norm(a.p-c.p),b.radius+norm(b.p-c.p));
  c.R = Rotation();
  c.m = (a.m+b.m);
  c.p = (a.m*a.p + b.m*b.p)/c.m;
  c.K = a.K+b.K;
  c.L = a.L+cross(a.p-c.p,a.K)+b.L+cross(b.p-c.p,b.K);
  Vector da = a.p-c.p;
  Vector db = b.p-c.p;
  int na = a.r.size();
  int nb = b.r.size();
  c.I(X,X) = c.I(Y,Y)= c.I(Z,Z) =0;
  // copy all r
  Vector v;
  c.r.resize(na+nb);
  for(int i=0; i<na; i++) {
    v = a.R*a.r[i]+da;
    c.r[i] = v;
    c.I = c.I + dI(a.m/na,v);
  }
  for(int i=0; i<nb; i++) {
    v = b.R*b.r[i]+db;
    c.r[i+na] = v;
    c.I = c.I + dI(b.m/nb,v);
  }
  // copy all faces and re-label r
  int m = a.faces.size();
  c.faces.resize(a.faces.size()+b.faces.size());
  for(int j=0; j<a.faces.size(); j++)
    c.faces[j]=a.faces[j];
  for(int j=0; j<b.faces.size(); j++)
    for(int k=0; k<b.faces[j].size(); k++)
      c.faces[j+m].push_back(b.faces[j][k]+na);
  c.inv_I = 1.0/c.I;
  c.update_vertices();
  return c;
}
\end{lstlisting}
\noindent
Function that loads an wavefront obj file into a body: \begin{lstlisting}
void Body::loadObj(const string & file,float scale=0.5) {
  ifstream input;
  string line;
  float x,y,z;
  r.resize(0);
  faces.resize(0);
  input.open(file.c_str());
  if(input.is_open()) {
    while(input.good()) {
      getline(input, line);
      if(line.length()>0) {
        string initialVal;
        istringstream instream;
        instream.str(line);
        instream >> initialVal;
        if(initialVal=="v") {
          instream >> x >> y >> z;
          Vector p = scale*Vector(x,y,z);
          r.push_back(p);
          radius = max(radius,norm(p));
        } else  if (initialVal=="f") {
          array<int> path;
          while(instream >> x) path.push_back(x-1);
          faces.push_back(path);
        }
      }
    }
    update_vertices();
  }
}
\end{lstlisting}
\noindent
Make a universe with an airplane: \begin{lstlisting}
class MyUniverseAirplane : public Universe {
  Body plane;
public:
  void build_universe() {
    plane.color=Vector(1,0,0); //red
    plane.loadObj("assets/plane.obj",0.5);    
    plane.R = Rotation(Vector(0,Pi,0));
    bodies.insert(&plane);    
    forces.insert(new GravityForce(&plane,0.5));
    // constraints.insert(new PlaneConstraint(&plane,0.0,Vector(0,-0.2,0),Vector(0,-1,0)));
  }
  void callback() {
  }
  void keyboard(unsigned char key, int x, int y) {
    if(key=='w') plane.L = plane.L + Vector(+0.1,0,0);
    if(key=='a') plane.L = plane.L + Vector(0,0,+0.1);
    if(key=='d') plane.L = plane.L + Vector(0,0,-0.1);
    if(key=='z') plane.L = plane.L + Vector(-0.1,0,0);
    if(key=='n') plane.K = plane.K + Vector(0,0,+1);
    if(key=='m') plane.K = plane.K + Vector(0,0,-1);    
  }
};


array<float> random_vector(int n, float M) {
  float one_norm = 0.0;
  array<float> m(n);
  for(int i=0; i<n; i++) { m[i] = uniform(); one_norm += m[i]; }
  for(int i=0; i<n; i++) m[i]=M*m[i]/one_norm;
  return m;
}
\end{lstlisting}
\noindent
Make a universe with an exploding body: \begin{lstlisting}
class SimpleUniverse : public Universe {
private:
  Body* xb;
public:
  void build_universe() {
    float m = 1.0;
    float v = 5.0;
    float theta = Pi/2.2;
    Body &b = *new Body(m);
    b.color = Vector(uniform(),uniform(),uniform()); //red
    b.loadObj("assets/sphere.obj",0.5);
    b.p = Vector(0,10,-10);
    b.K = Vector(0,0,0);
    b.L = Vector(0,0,0);
    bodies.insert(&b);
    forces.insert(new GravityForce(&b,1.0));
    xb = &b;
  }
  void callback() {
    if(frame==150) {
      int n = 50;
      float vx,vy,vz;
      float E = 0.5;
      float mysum_x=0, mysum_y=0, mysum_z=0;
      array<float> m = random_vector(n,xb->m);
      array<float> eps_x(n), eps_y(n), eps_z(n);
      xb->visible = false;
      for(int j=0; j<n; j++) {
        cout << j << "  " << uniform() << endl;
        Body &a = *new Body(m[j]);
        a.color = Vector(uniform(),uniform(),uniform());
        a.p = Vector(xb->p(X), xb->p(Y), xb->p(Z));
        if(j<n-1) {
          eps_x[j] = uniform(-1,1)*sqrt(E);
          eps_y[j] = uniform(-1,1)*sqrt(E);
          eps_z[j] = uniform(-1,1)*sqrt(E);
          mysum_x += m[j]*eps_x[j];
          mysum_y += m[j]*eps_y[j];
          mysum_z += m[j]*eps_z[j];
        } else {
          eps_x[j] = -mysum_x/m[j];
          eps_y[j] = -mysum_y/m[j];
          eps_z[j] = -mysum_z/m[j];
        }
        vx = xb->v(X) + eps_x[j];
        vy = xb->v(Y) + eps_y[j];
        vz = xb->v(Z) + eps_z[j];
        a.K = Vector(m[j]*vx, m[j]*vy, m[j]*vz);
        bodies.insert(&a);
        forces.insert(new GravityForce(&a,1.0));
      }
    }

    float restitution = 0.5;
    forEach(ibody,bodies)

      if(OBJ(ibody).p(Y)<0 && OBJ(ibody).K(Y)<0) {
        OBJ(ibody).p(Y)=-restitution*OBJ(ibody).p(Y);
        OBJ(ibody).K(Y)=-restitution*OBJ(ibody).K(Y);
      }
  }
};
\end{lstlisting}
\noindent
Make a universe with some bouncing balls: \begin{lstlisting}
class MyUniverse : public Universe {
public:
  void build_universe() {
    Body *b_old = 0;
    for(int i=0; i<30; i++) {
      Body &b = *new Body();
      b.color=Vector(uniform(0,1),uniform(0,1),uniform(0,1));
      b.loadObj("assets/sphere.obj");
      b.p = Vector(uniform(-2,2),uniform(1,3),uniform(-2,2));
      b.K = Vector(uniform(-2,2),uniform(-2,2),0);
      b.L = Vector(uniform(0,1),uniform(0,1),uniform(0,1));
      bodies.insert(&b);
      forces.insert(new GravityForce(&b,1));
      constraints.insert(new PlaneConstraint(&b,0.9,Vector(0,0,0),
                                             Vector(0,-1,0)));
      constraints.insert(new PlaneConstraint(&b,0.9,Vector(0,5,0),
                                             Vector(0,+1,0)));
      constraints.insert(new PlaneConstraint(&b,0.9,Vector(5,0,0),
                                             Vector(1,0,0)));
      constraints.insert(new PlaneConstraint(&b,0.9,Vector(-5,0,0),
                                             Vector(-1,0,0)));
      constraints.insert(new PlaneConstraint(&b,0.9,Vector(0,0,5),
                                             Vector(0,0,1)));
      constraints.insert(new PlaneConstraint(&b,0.9,Vector(0,0,-5),
                                             Vector(0,0,-1)));
    }
    constraints.insert(new All2AllCollisions(&bodies,0.8));
    // forces.insert(new Water(&bodies,3.0));
  }
};
\end{lstlisting}
\noindent
Make a universe with composite objects made of cubes: \begin{lstlisting}
class MyUniverseCubes : public Universe {
public:
  Body cube(const Vector &p, const Vector& theta, const Vector &color) {
    Body cube;
    cube.color = color;
    cube.p = p;
    cube.R = Rotation(theta);
    for(int x=-1; x<=+1; x+=2)
      for(int y=-1; y<=+1; y+=2)
        for(int z=-1; z<=+1; z+=2)
          cube.r.push_back(Vector(x,y,z));
    cube.faces.resize(6);
    for(int i=0; i<2; i++) {
      cube.faces[i].push_back(0+4*i);
      cube.faces[i].push_back(1+4*i);
      cube.faces[i].push_back(3+4*i);
      cube.faces[i].push_back(2+4*i);
      cube.faces[i+2].push_back(0+2*i);
      cube.faces[i+2].push_back(1+2*i);
      cube.faces[i+2].push_back(5+2*i);
      cube.faces[i+2].push_back(4+2*i);
      cube.faces[i+4].push_back(0+i);
      cube.faces[i+4].push_back(2+i);
      cube.faces[i+4].push_back(6+i);
      cube.faces[i+4].push_back(4+i);
    }
    cube.update_vertices();
    return cube;
  }
  void build_universe() {
    Body *thing = new Body();
    (*thing) =
      cube(Vector(0,4,0),Vector(0,1,0),Vector(0,1,0)) +
      cube(Vector(1,4,0),Vector(1,0,0),Vector(0,1,0.2)) +
      cube(Vector(1,4.5,.5),Vector(1,0,1),Vector(0.2,1));
    (*thing).L = Vector(0,0.1,0);
    bodies.insert(thing);
    Body *other_thing = new Body();
    (*other_thing) = (*thing);
    (*other_thing).p = Vector(-1,3,0);
    (*other_thing).L = Vector(-0.1,0.0,0.1);
    bodies.insert(other_thing);
  }
};

class CompositionUniverse: public Universe {
private:
  Body b;
public:
  void build_universe() {
    Body a;
    float x,y,z;
    for(int i=0; i<8; i++) {
      x = 2*((i>>0)&1)-1;
      y = 2*((i>>1)&1)-1;
      z = 2*((i>>2)&1)-1;
      a.p = Vector(x,y+1,z-5);
      if(i==0) b = a; else b=b+a;
    }
    b.L = Vector(5,10,0);
    bodies.insert(&b);
  }
};

MyUniverse universe;
// MyUniverseAirplane universe;
// MyUniverseCubes universe;
// SimpleUniverse universe;
// CompositionUniverse universe;
\end{lstlisting}
\noindent
Glut code below
creates a window in which to display the scene: \begin{lstlisting}
void createWindow(const char* title) {
  int width = 640, height = 480;
  glutInitDisplayMode(GLUT_DOUBLE | GLUT_RGB | GLUT_DEPTH);
  glutInitWindowSize(width,height);
  glutInitWindowPosition(0,0);
  glutCreateWindow(title);
  glClearColor(0.9f, 0.95f, 1.0f, 1.0f);
  glEnable(GL_DEPTH_TEST);
  glShadeModel(GL_SMOOTH);
  glMatrixMode(GL_PROJECTION);
  glLoadIdentity();
  gluPerspective(60.0, (double)width/(double)height, 1.0, 500.0);
  glMatrixMode(GL_MODELVIEW);
}
\end{lstlisting}
\noindent
Called each frame to update the 3d scene. delegates to
the application: \begin{lstlisting}
void update(int value) {
  // evolve world
  universe.dt = DT;
  universe.evolve();
  glutTimerFunc(MSPF, update, 0);
  glutPostRedisplay();
}
\end{lstlisting}
\noindent
Function called each frame to display the 3d scene.
it draws all bodies, forces and constraints: \begin{lstlisting}
void display() {
  glClear(GL_COLOR_BUFFER_BIT | GL_DEPTH_BUFFER_BIT);
  glLoadIdentity();
  gluLookAt(0.0,3.5,10.0, 0.0,3.5,0.0, 0.0,1.0,0.0);
  forEach(universe.ibody,universe.bodies)
    if(OBJ(universe.ibody).visible)
                OBJ(universe.ibody).draw();
  forEach(universe.iforce,universe.forces)
    OBJ(universe.iforce).draw();
  forEach(universe.iconstraint,universe.constraints)
    OBJ(universe.iconstraint).draw();
  // update the displayed content
  glFlush();
  glutSwapBuffers();
}
\end{lstlisting}
\noindent
Code that draws an body: \begin{lstlisting}
void Body::draw() {
  glPolygonMode(GL_FRONT,GL_FILL);
  if(faces.size()==0) {
    glColor3f(color(0),color(1),color(2));
    int n = vertices.size();
    for(int i=0; i<n; i++) {
      glPushMatrix();
      glTranslatef(vertices[i].v[X],vertices[i].v[Y],vertices[i].v[Z]);
      glutSolidSphere(0.2,10,10);
      glPopMatrix();
    }
  } else
    for(int i=0; i<faces.size(); i++) {
      float k = 0.5*(1.0+(float)(i+1)/faces.size());
      glColor3f(color(0)*k,color(1)*k,color(2)*k);
      glBegin(GL_POLYGON);
      for(int j=0; j<faces[i].size(); j++)
        glVertex3fv(vertices[faces[i][j]].v);
      glVertex3fv(vertices[faces[i][0]].v);
      glEnd();
    }
}
\end{lstlisting}
\noindent
Code that draws a spring: \begin{lstlisting}
void SpringForce::draw() {
  glColor3f(0,0,0);
  glPolygonMode(GL_FRONT_AND_BACK,GL_LINE);
  glBegin(GL_LINES);
  if(iA<0)  glVertex3fv(bodyA->p.v);
  else glVertex3fv(bodyA->vertices[iA].v);
  if(iB<0)  glVertex3fv(bodyB->p.v);
  else glVertex3fv(bodyB->vertices[iB].v);
  glEnd();
}
\end{lstlisting}
\noindent
Draw the the water: \begin{lstlisting}
void Water::draw() {
  glPolygonMode(GL_FRONT,GL_FILL);
  for(int i=0; i<40; i++) {
    glBegin(GL_POLYGON);
    for(int j=0; j<40; j++) {
      glColor3f(0,0,0.5+0.25*(m[i][j]-level+wave)/wave);
      glVertex3fv(Vector(-x0+dx*i,m[i][j],-x0+dx*j).v);
      glVertex3fv(Vector(-x0+dx*i,m[i][j+1],-x0+dx*j+dx).v);
      glVertex3fv(Vector(-x0+dx*i+dx,m[i+1][j+1],-x0+dx*j+dx).v);
      glVertex3fv(Vector(-x0+dx*i+dx,m[i+1][j],-x0+dx*j).v);
    }
    glEnd();
  }
}
\end{lstlisting}
\noindent
Function called when the display window changes size: \begin{lstlisting}
void reshape(int width, int height) {
  glViewport(0, 0, width, height);
}
\end{lstlisting}
\noindent
Function called when a mouse button is pressed: \begin{lstlisting}
void mouse(int button, int state, int x, int y) { 
  universe.mouse(button,state,x,y);
}
\end{lstlisting}
\noindent
Function called when a key is pressed: \begin{lstlisting}
void keyboard(unsigned char key, int x, int y) {
  universe.keyboard(key,x,y);
}
\end{lstlisting}
\noindent
Called when the mouse is dragged: \begin{lstlisting}
void motion(int x, int y) { }
\end{lstlisting}
\noindent
The main function. everythign starts here: \begin{lstlisting}
int main(int argc, char** argv) {
  // Create the application and its window
  glutInit(&argc, argv);
  createWindow("Cylon");
  // fill universe with stuff
  universe.build_universe();
  // Set up the appropriate handler functions
  glutReshapeFunc(reshape);
  glutKeyboardFunc(keyboard);
  glutDisplayFunc(display);
  glutMouseFunc(mouse);
  glutMotionFunc(motion);
  glutTimerFunc(MSPF, update, 0); // Do not refresh faster than target framerate
  // Enter into a loop
  glutMainLoop();
  return 0;
}
\end{lstlisting}
